\documentclass{book}

\usepackage{amsmath,amssymb,amsfonts}

\usepackage{fouche}
\usepackage[esperanto]{babel}

\author{Fouche}

\usepackage{newtxtext,newtxmath}
\title{Teorio de kategorioj}
\begin{document}
\maketitle

\chapter{Bazaj difinoj}
\begin{definition}[Kategorio]
  % Una categoria è una terna $(s,t,c)$ di funzioni tra classi
  \[\xymatrix{\clC_0 \ar[r] & \clC_1\ar@<4pt>[l] \ar@<-4pt>[l]& \ar[l] \clC_1 \times_{\clC_0} \clC_1}\]
\end{definition}
\begin{definition}[Functo]
  
\end{definition}
\begin{definition}[Natura transformo]
  Estu $\clC,\clD$ du kategorioj; tiam natura transformo $\alpha : F \Rightarrow G$ estas kolekto de sagoj $\alpha : Fc \to Gc$ tia ke por ĉiu sago $f : c \to c'$ en $\clC$ ĉi tiu kvadrato komutas en $\clD$:
\end{definition}
\chapter{Universalaj konstruejoj}
\chapter{Monaoj kaj iliaj algebroj}
\end{document}